% TODO: P4
% Intro: Tell the full story of your paper at a high-level. I like the hourglass approach - start broad to appeal to the general audience, narrow into your specific approach, proposal and idea and finally end with a discussion on how the work fits into the grand scheme of things.
\section{Introduction}
\textcolor{black!80}{Large language models (LLMs) are increasingly engaged as conversational partners in personal domains, offering users not only informational guidance but also affective support \cite{zhang-ai_companions, phang-affective, anthropic2025}. Their appeal lies in features such as anonymity, immediacy, and the absence of social risk--qualities shared with online communities like Reddit. Yet, unlike human interlocutors, LLMs lack grounding in lived social contexts, raising critical questions about how their responses should be evaluated and trusted in a social context.}

\textcolor{black!80}{Emerging research often identifies two contrasting tendencies in LLM outputs in isolation: empathic responses resembling desirable and supportive therapeutic dialogue, and sycophantic ones that uncritically echo a user’s perspective. Whether such responses are judged as empathic or sycophantic can depend on the psychosocial framework applied. This ambiguity underscores a critical gap: systematic methods are needed to analyze the responses and compare them to human written ones. This project uses the latter as proxies for normative ground truths, providing a measurement of these behaviors and values across the different psychosocial paradigms.}

\textcolor{black!80}{The comparisons made are across Rogerian person-centered therapy (PCT), Goffman’s theory of face (ToF), and Rokeach’s Value Survey (RVS) framework. The framework is designed to be extensible, allowing researchers to incorporate additional paradigms as the field evolves. Additionally, we use the insights from these analyses to inform the generation of customized responses with chain-of-thought control mechanisms.}
% Extend to Supervised fine-tuning given the time.


% 2. GPT suggestion: Narrowing to gaps -> then RQs
\subsection{Research Questions}\label{sec:RQs}
The context of queries can substantially shape LLM outputs, influencing not only personal questions posed by consumers but also analytical evaluations conducted by researchers, particularly within the LLM-as-a-judge paradigm. As research increasingly highlights patterns and concerns regarding the impacts of LLMs in personal queries and deliberation, there is a critical need for a framework that can analyze and compare responses across multiple value-based perspectives in contexts without clear normative answers, while also remaining extensible for researchers to incorporate additional paradigms as the field evolves. This motivates the following research questions: 

\medskip\textbf{RQ1:} How can a technical framework that systematically analyzes and compares responses from humans and LLMs across various psychosocial value paradigms be designed?  

\medskip\textbf{RQ2:} What inter- and intra-paradigm comparative insights can this framework yield across four different psychosocial frameworks and how accurate are these?
\textbf{RQ2a:} To what extent can identical features be annotated with divergent connotations across paradigms—empathic under Rogerian PCT versus sycophantic under Goffman’s ToF?

\medskip\textbf{RQ3:} How do LLM-generated responses compare to human-authored responses in the context of personal questions without definitive normative answers?

Finally, we examine how the results may come to influence consumer behavior and broader societal outcomes. We explore a potential control mechanism with Chain of Thought (CoT) reasoning. Our work enables a systematic comparative analysis of potential benefits and risks, and presents a framework for analysis which can be used by researchers and consumers for leveraging the insights in the intentional design of response LLM generation.


% One researcher's question was around the values in the OP's body of the post and the values in the people's response. Focus on some of the major influences in LLM contexts and how can they be used to influence the generation of responses with DeepReflect's analyses ? This can be gracefully woven into the discussion / future investigations section although a reference could be made here if necessary - demonstrating the systems' potential to address questions of researchers' investigative curiosity.

% 3. The contributions as "responses" to the RQs - but I'd like to pose that subtly.

% \textcolor{black!30}{Finally, we provide a demonstration of DeepReflect's capabilities in the generation of responses to personal queries - contrasting them with raw LLM responses.
% Several other research questions can be addressed with the design of the system - aiding the motivation of researchers and providing an improved design to users. While a discussion on these is provided in the Conclusion section (Section~\ref{sec:Conclusion}), the focus of this paper is on the three research questions above owing to constraints on time and other resources.}

% \subsection{Contributions}
% \textcolor{black!30}{Refer to the Sycophancy paper for how to write this.Reference discussion section in Conclusion (section 7) to broaden back out - how this work can fit in the grand scheme of LLM conversations and research.}

\subsection{Contributions}
The key contributions of this work are: (1) the design and implementation of an extensible framework for analyzing and comparing responses to personal queries across three distinct psychosocial paradigms; (2) a comparative analysis under Rogerian Person-Centered Therapy (PCT), Goffman’s theory of face and Rokeach’s Value Survey (RVS) framework, illustrating how the choice of the paradigm can shape the perception of a response; and (3) insights into the relative strengths and weaknesses of LLM versus human responses, and how these insights can inform the generation of customized responses to personal queries.

% \begin{itemize}
%     \item \textcolor{black!30}{\lipsum[8]}
%     \item \textcolor{black!30}{\lipsum[9]}
%     \item \textcolor{black!30}{\lipsum[10]}
% \end{itemize}