% TODO: P7
% Analysis
% Discussion of what the results mean, what they don't mean, where they can be improved, etc. These sections vary a lot depending on the nature of the paper.
% For papers reporting on experiments with multiple datasets, it can be good to repeat Methods/Results/Analysis in separate (sub)sections for each dataset.

\section{Analysis}\label{sec:Analysis}

\textcolor{black!40}{Discussion of what the results mean, what they don't mean, where they can be improved, etc. These sections vary a lot depending on the nature of the paper.For papers reporting on experiments with multiple datasets, it can be good to repeat Methods/Results/Analysis in separate (sub)sections for each dataset.}

The \LaTeX{} and Bib\TeX{} style files provided roughly follow the American Psychological Association format.
If your own bib file is named \texttt{custom.bib}, then placing the following before any appendices in your \LaTeX{} file will generate the references section for you:
\begin{quote}
\begin{verbatim}
\bibliographystyle{acl_natbib}
\bibliography{custom}
\end{verbatim}
\end{quote}
% You can obtain the complete ACL Anthology as a Bib\TeX{} file from \url{https://aclweb.org/anthology/anthology.bib.gz}.
% To include both the Anthology and your own .bib file, use the following instead of the above.
% \begin{quote}
% \begin{verbatim}
% \bibliographystyle{acl_natbib}
% \bibliography{anthology,custom}
% \end{verbatim}
% \end{quote}

\subsection{Interpretation of Results}
\textcolor{black!30}{\lipsum[53-56]}

\subsection{Theoretical Implications}
\textcolor{black!30}{\lipsum[57-59]}
\nocite{Ando2005,augenstein-etal-2016-stance,andrew2007scalable,rasooli-tetrault-2015,goodman-etal-2016-noise,harper-2014-learning}
\subsection{Subsection}
The framework is capable of producing several informative plots of research interest. One such summary plot is a heatmap showcasing the values exhibited in the OPs post against the responses to support the investigation of several other potential research questions in this theme of interest (discussed in the future work section).
\textcolor{black!30}{\lipsum[60-62]}