% TODO: P3
% Methods
% Methods: The experimental approach, including descriptions of metrics, baseline models, etc. Details about hyperparameters, optimization choices, etc., are probably best given in appendices, unless they are central to the arguments.

\section{Methods} \label{sec:Methods}
% ~1200 words ≈ ~9–10 paragraphs
\textcolor{black!30}{\lipsum[11-12]}

\subsection{Data collection and preprocessing}
% This section can describe text processing steps - how the human responses were selected and which posts were selected from the RedArcs dataset etc.
We built a corpus from two public subreddits—AITAH, and Anxiety. For each subreddit, we filtered the top ~1,000 most upvoted or most commented posts, excluding weekly megathreads, deleted/removed items, and AutoModerator entries. For every retained post we extracted (i) the most upvoted human-written comment and (ii) the comment that the OP engaged with most; all artifacts were saved to standardized CSVs for downstream analysis.

Text was cleaned with minimal, semantics-preserving preprocessing: we removed non-English items, de-identified obvious personal identifiers (usernames, emails, links), standardized whitespace and Unicode characters, and lightly constrained length (posts ~50–500 words; comments ~5–300 words) for comparability. We treat each Reddit thread (the post and its comments) as a single analytic unit during sampling, manual checks, and statistic aggregation, so correlated texts don’t inflate results. This preserves thread integrity and prevents dependence-induced bias when comparing human and LLM responses drawn from the same conversation.We also removed exact and near-duplicate texts (specifically, crossposts, copypastes and bot repeats) to prevent inflated counts and biased comparisons. 

\smallskip Prompts (provided in the appendices) and model outputs are saved and logged by the codebase for reproducibility.

\subsection{Experiments}
\textcolor{black!30}{\lipsum[12-14]}

\subsubsection{Response Generation}
\textcolor{black!40}{This section is critical for the reader so they can piece together where the analyses fit in with the rest of the paper - how they can be used for generation. A short section describing how the distribution of values can be used to generate responses.}

\subsection{Measurement}
\textcolor{black!30}{\lipsum[14-15]}

\subsection{Construct Validity}
TODO later.

\subsection{Evaluation Metrics}
TODO later.




% Users of older versions of \LaTeX{} may encounter the following error during compilation: 
% \begin{quote}
% \tt\verb|\pdfendlink| ended up in different nesting level than \verb|\pdfstartlink|.
% \end{quote}
% This happens when pdf\LaTeX{} is used and a citation splits across a page boundary. The best way to fix this is to upgrade \LaTeX{} to 2018-12-01 or later.