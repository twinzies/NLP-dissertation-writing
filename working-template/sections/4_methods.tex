% TODO: P3
% Methods
% Methods: The experimental approach, including descriptions of metrics, baseline models, etc. Details about hyperparameters, optimization choices, etc., are probably best given in appendices, unless they are central to the arguments.

\section{Methodology}\label{sec:Methodology}
% ~1200 words ≈ ~9–10 paragraphs
\textcolor{black!30}{Methods: The experimental approach, including descriptions of metrics, baseline models, etc. Details about hyperparameters, optimization choices, etc., are probably best given in appendices, unless they are central to the arguments.}

\subsection{Data collection and preprocessing}
% This section can describe text processing steps - how the human responses were selected and which posts were selected from the RedArcs dataset etc.
All selected posts were manually reviewed to ensure that no sensitive or private information was included. 
\textcolor{black!30}{\lipsum[30-33]}

\subsection{Experiments}
\textcolor{black!30}{\lipsum[12-14]}

\subsubsection{Response Generation}
\textcolor{black!40}{This section is critical for the reader so they can piece together where the analyses fit in with the rest of the paper - how they can be used for generation. A short section describing how the distribution of values can be used to generate responses.}

\subsection{Measurement}
\textcolor{black!30}{\lipsum[14-15]}

\subsection{Construct Validity}
TODO later.

\subsection{Evaluation Metrics}
TODO later.




% Users of older versions of \LaTeX{} may encounter the following error during compilation: 
% \begin{quote}
% \tt\verb|\pdfendlink| ended up in different nesting level than \verb|\pdfstartlink|.
% \end{quote}
% This happens when pdf\LaTeX{} is used and a citation splits across a page boundary. The best way to fix this is to upgrade \LaTeX{} to 2018-12-01 or later.