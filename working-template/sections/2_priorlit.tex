% TODO: P1
% Prior literature: Contextualize your work and provide insights into major relevant themes of the literature as a whole. Use each paper (or theme) as a chance to articulate what is special about your paper.
\section{Prior Literature}\label{sec:Litreview}
% ~1440 words ≈ ~11–12 paragraphs

\textcolor{black!40}{Contextualize your work and provide insights into major relevant themes of the literature as a whole. Use each paper (or theme) as a chance to articulate what is special about your paper. Start out broad - social background and theory - Discuss what other frameworks were considered like Virtue ethics and philosophical ones, CBT, Schwartz values etc. but why they were not chosen. Why I Focused on Rogerian psychotherapy as it is person centered - no specific diagnosis needed (or available).}


\subsection{Theoretical Foundations}

\subsection{Rogerian Psychotherapy}
\textcolor{black!30}{\lipsum[8-9]}
\subsubsection{Psychosocial use and Empathic LLMs}
\textcolor{black!30}{\lipsum[14-16]}
\textcolor{black!30}{Katie mentioned a good point about how I'm adding greater nuance to the Likert scales referenced in this paper.}

\subsection{Rokeach Value Survey as an analytical instrument}
\textcolor{black!30}{\lipsum[6-7]}
\subsubsection{Values and Ethics in LLM research}
\textcolor{black!30}{\lipsum[7-8]}
\textcolor{black!30}{Add some notes and mention how Anthropic's work warrants some scrutiny as they are a for-profit company. The "values" framework they propose in values in the wild has not been validated by experts in the social sciences. However it provides a good reference frame for comparison with the Rokeach framework of values. There is a limitation - DeepReflect does not have access to the full dataset Anthropic used for the Values in the Wild paper.}

% Narrow down to LLMs and trends in the new approaches are emerging - commonalities in the trends.
% social engineering

\subsection{Goffman's theory of face}
\textcolor{black!30}{\lipsum[9-10]}
\subsubsection{Social Sycophancy in LLMs}
% \textcolor{black!30}{Recent research has shown that language models often exhibit sycophantic behavior, agreeing with users regardless of statement accuracy \cite{sharma2023understanding}.}
\textcolor{black!30}{I already have lots of good notes on this in writing.}
\textcolor{black!30}{\lipsum[12-14]}

\subsection{Gaps in the Literature and Open Challenges}
In sum, as LLM-chatbots have become increasingly human-like and more users seek companionship with them, studies have highlighted both the advantages and disadvantages of their use. While some have raised concerns around “emotional dependence” \cite{fang-etal-psychoeffects}, several others have explored empathic perceptions of LLM responses and found them advantageous not only in the field of medical support and mental health but also in everyday personal queries \cite{Lee-etal-Empathic, sorin-etal-empathy}.
However, different psychosocial paradigms tend to frame LLM responses in markedly divergent terms. \textbf{What may be perceived as ‘empathy’} under a psychotherapeutic paradigm could \textbf{instead be critiqued as an instance of ‘social sycophancy’} by frameworks informed by Goffman's Theory of Face \cite{cheng-etal-sycophancy}.
Importantly, in the absence of clear normative answers, the same statement may be categorised as ‘face-preserving behaviour’ or ‘unconditional positive regard’. 


DeepReflect provides a comparative framework to address this gap by assessing how evaluative judgments are shaped by the psychosocial paradigm through which a response is framed. 
Moreover, it is designed to be extensible by researchers, enabling the incorporation of both conventional paradigms, such as Rokeach’s values framework, and contemporary discovery-based approaches, such as Anthropic’s Values in the Wild \cite{values-in-wild}, whereas prior work has tended to focus on a single paradigm in isolation.

\medskip Finally, our investigation of controlling generations avoids replicating prior work that seeks to mitigate sycophancy exclusively \cite{cheng-etal-sycophancy}. Instead of treating sycophancy as a defect to be eliminated in isolation, DeepReflect provides a system to situate response generation within extensible psychosocial frameworks. This ensures that outputs are not merely reactive to user prompts but can be guided by well-established instruments for values and personal-growth. 


In practice, this involves chain-of-thought reasoning \cite{cot} that explicitly incorporates the chosen framework. Unlike approaches that mimic deliberation across hypothetical perspectives \cite{socialgaze}, this control strategy extends the contractualist, rule-based tradition of questioning developed in \cite{cot-morals}. Its key distinction lies in embedding the questioning within expert-informed guidelines. While these prior investigations emphasized plurality of viewpoints and normative exception-handling, this work foregrounds the role of pre-existing psychosocial instruments in shaping the ongoing, ever-changing conversations of personal reflection.