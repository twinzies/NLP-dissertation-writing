% TODO: P8
% Quickly summarize what the paper did, and then chart out possible future directions that anyone might pursue. Finish with a strong conclusion.
% Avoid subjective wording such as "unprecedented", "padeoning", or "groundbreaking".

\section{Conclusion}\label{sec:Conclusion}
/textcolor{black!40}{Quickly summarize what the paper did, and then chart out possible future directions that anyone might pursue. Finish with a strong conclusion. Avoid subjective wording such as "unprecedented", "pioneering", or "groundbreaking".}
\subsection{Summary of Findings}
\textcolor{black!30}{\lipsum[63-64]}

% The Presentation rubrics define the following criteria: 
% Conclusions should be competently covered and Further work not to be overly general.
% I am aiming for the conclusions drawn to be correct and show ability to summarise with acumen. The discussion of future work is appropriate to the evaluation of work done.
\subsubsection{Discussion}
Epistemic limits in interpreting behavior through psychosocial theories are not unique to LLMs but are equally present in human communication. Recent advances in NLP provide opportunities to systematically translate qualitative theories into quantitative analyses, thereby enabling a more rigorous investigation of these epistemic limits. Nevertheless, this remains an open challenge that extends beyond the scope of NLP research and requires engagement from the broader social science and humanities communities.
It would be misleading to assume that an observed feature is purely “sycophantic” or “empathic” without due consideration for the context of the personal interaction and the needs of the individual.
\subsection{Future Directions}
\textcolor{black!30}{\lipsum[65]}

% Talk about the alignment problem here and how the chain of thought reasoning with psychosocial frameworks can help address the situationally aware reward hacking issue or be used for political propaganda if we are not evaluating the models with frameworks like DeepReflect.
% Katie made a good point referencing the Online Safety Amendment Act 2024 (Australia) and the online safety bill in the UK which could exacerbate loneliness for minority groups that find themselves to be psycologically isolated without restricting full access to social media - DeepReflect could provide a channel to address this issue - without restricting the full access to the benefits of technology for the groups affected by such new changes in legislation. [Cite the MIT study on psychosocial impact of chatbot use - tying it back to the literature].

% Scientific work published at ACL 2023 must comply with the ACL Ethics Policy.\footnote{\url{https://www.aclweb.org/portal/content/acl-code-ethics}} We encourage all authors to include an explicit ethics statement on the broader impact of the work, or other ethical considerations after the conclusion but before the references. The ethics statement will not count toward the page limit (8 pages for long, 4 pages for short papers).