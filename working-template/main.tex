% Stuart Shieber on the Rational Reconstruction style of paper
% The ideal style is the "rational reconstruction" style. In this style, you don't present the actual history that you went through, but rather an idealized history that perfectly motivates each step in the solution. "We consider the problem of XXX. The obvious thing to try is X. But such-and-such a pithy example shows that that fails miserably. Nonetheless, the example points the way naturally to solution Y. This works better, except for such-and-such an obscure case. We patch solution Y to handle this case, forming solution Z. Voila." 

% Of course, the author doesn't tell you that he came up with solution Y before solution X, which only occurred to him after he came up with solution Z, and he skips solutions A, B, and C because, in retrospect, they are nowhere on the natural path to Z, even though at the time he was completely convinced they were on the right track. 

% The goal in pursuing the rational reconstruction style is not to convince the reader that you are brilliant (or addle-headed for that matter) but that *your solution is trivial*. It takes a certain strength of character to take that as one's goal. But the advantage of the reader thinking your solution is trivial or obvious is that it necessarily comes along with the notion that *you are correct*.

% "A good talk should never stray far from simple, honest communication."

\pdfoutput=1
\documentclass[11pt]{article}
\usepackage[review]{ACL2023}
\usepackage{xcolor} % To make the lipsum text grey.
\usepackage{lipsum}
% Standard package includes
\usepackage{times}
\usepackage{latexsym}
\usepackage[T1]{fontenc}
\usepackage[utf8]{inputenc}
\usepackage{microtype}
\usepackage{inconsolata}
\usepackage{graphicx} % Required for including images
\usepackage{amsmath}

% Uncomment and adjust if title/author block needs more space
% \setlength\titlebox{5cm}

\title{“Sycophancy” or “Empathy”? \\
DeepReflect – An LLM-based system designed to analyze and generate responses to personal queries 
}

\author{Tara K. Jain \\
  Affiliation / Address line 1 \\
  Affiliation / Address line 2 \\
  Affiliation / Address line 3 \\
  \texttt{email@domain} \\\And
  Diana Maynard \\
  Affiliation / Address line 1 \\
  Affiliation / Address line 2 \\
  Affiliation / Address line 3 \\
  \texttt{email@domain} \\}

\begin{document}
\maketitle
\begin{abstract}
% TODO: P5
Large language models (LLMs) are increasingly used for personal queries, recent research has involved analyzing responses under psychosocial framing. This work introduces DeepReflect, a comparative framework for analyzing human and AI generated responses to personal queries across multiple paradigms of values and social behavior.
\textcolor{black!30}{\lipsum[2-2]}
\end{abstract} 
% Intro: Tell the full story of your paper at a high-level. I like the hourglass approach - start broad to appeal to the general audience, narrow into your specific approach, proposal and idea and finally end with a discussion on how the work fits into the grand scheme of things.
\section{Introduction}

\lipsum[1-6]

\subsection{Contributions}
\lipsum[6-7]
% TODO: P1
% Prior literature: Contextualize your work and provide insights into major relevant themes of the literature as a whole. Use each paper (or theme) as a chance to articulate what is special about your paper.
\section{Prior Literature}\label{sec:Litreview}
% ~1440 words ≈ ~11–12 paragraphs

\textcolor{black!40}{Contextualize your work and provide insights into major relevant themes of the literature as a whole. Use each paper (or theme) as a chance to articulate what is special about your paper. Start out broad - social background and theory - Discuss what other frameworks were considered like Virtue ethics and philosophical ones, CBT, Schwartz values etc. but why they were not chosen. Why I Focused on Rogerian psychotherapy as it is person centered - no specific diagnosis needed (or available).}


\subsection{Theoretical Foundations}

\subsection{Rogerian Psychotherapy}
\textcolor{black!30}{\lipsum[8-9]}
\subsubsection{Psychosocial use and Empathic LLMs}
\textcolor{black!30}{\lipsum[14-16]}
\textcolor{black!30}{Katie mentioned a good point about how I'm adding greater nuance to the Likert scales referenced in this paper.}

\subsection{Rokeach Value Survey as an analytical instrument}
\textcolor{black!30}{\lipsum[6-7]}
\subsubsection{Values and Ethics in LLM research}
\textcolor{black!30}{\lipsum[7-8]}
\textcolor{black!30}{Add some notes and mention how Anthropic's work warrants some scrutiny as they are a for-profit company. The "values" framework they propose in values in the wild has not been validated by experts in the social sciences. However it provides a good reference frame for comparison with the Rokeach framework of values. There is a limitation - DeepReflect does not have access to the full dataset Anthropic used for the Values in the Wild paper.}

% Narrow down to LLMs and trends in the new approaches are emerging - commonalities in the trends.
% social engineering

\subsection{Goffman's theory of face}
\textcolor{black!30}{\lipsum[9-10]}
\subsubsection{Social Sycophancy in LLMs}
% \textcolor{black!30}{Recent research has shown that language models often exhibit sycophantic behavior, agreeing with users regardless of statement accuracy \cite{sharma2023understanding}.}
\textcolor{black!30}{I already have lots of good notes on this in writing.}
\textcolor{black!30}{\lipsum[12-14]}

\subsection{Gaps in the Literature and Open Challenges}
In sum, as LLM-chatbots have become increasingly human-like and more users seek companionship with them, studies have highlighted both the advantages and disadvantages of their use. While some have raised concerns around “emotional dependence” \cite{fang-etal-psychoeffects}, several others have explored empathic perceptions of LLM responses and found them advantageous not only in the field of medical support and mental health but also in everyday personal queries \cite{Lee-etal-Empathic, sorin-etal-empathy}.
However, different psychosocial paradigms tend to frame LLM responses in markedly divergent terms. \textbf{What may be perceived as ‘empathy’} under a psychotherapeutic paradigm could \textbf{instead be critiqued as an instance of ‘social sycophancy’} by frameworks informed by Goffman's Theory of Face \cite{cheng-etal-sycophancy}.
Importantly, in the absence of clear normative answers, the same statement may be categorised as ‘face-preserving behaviour’ or ‘unconditional positive regard’. 


DeepReflect provides a comparative framework to address this gap by assessing how evaluative judgments are shaped by the psychosocial paradigm through which a response is framed. 
Moreover, it is designed to be extensible by researchers, enabling the incorporation of both conventional paradigms, such as Rokeach’s values framework, and contemporary discovery-based approaches, such as Anthropic’s Values in the Wild \cite{values-in-wild}, whereas prior work has tended to focus on a single paradigm in isolation.

\medskip Finally, our investigation of controlling generations avoids replicating prior work that seeks to mitigate sycophancy exclusively \cite{cheng-etal-sycophancy}. Instead of treating sycophancy as a defect to be eliminated in isolation, DeepReflect provides a system to situate response generation within extensible psychosocial frameworks. This ensures that outputs are not merely reactive to user prompts but can be guided by well-established instruments for values and personal-growth. 


In practice, this involves chain-of-thought reasoning \cite{cot} that explicitly incorporates the chosen framework. Unlike approaches that mimic deliberation across hypothetical perspectives \cite{socialgaze}, this control strategy extends the contractualist, rule-based tradition of questioning developed in \cite{cot-morals}. Its key distinction lies in embedding the questioning within expert-informed guidelines. While these prior investigations emphasized plurality of viewpoints and normative exception-handling, this work foregrounds the role of pre-existing psychosocial instruments in shaping the ongoing, ever-changing conversations of personal reflection.
% Your model: Flesh out your own approach, perhaps amplifying themes from the 'Prior lit' section.
% Data: Likely to be very detailed if the datasets are new or unfamiliar to the community, or if familiar datasets are being used in new ways.

\section{DeepReflect}
% ~1440 words ≈ ~11–12 paragraphs

\subsection{Framework}
\lipsum[8-10]

\subsection{Dataset}
\lipsum[10-12]
% Methods
% Methods: The experimental approach, including descriptions of metrics, baseline models, etc. Details about hyperparameters, optimization choices, etc., are probably best given in appendices, unless they are central to the arguments.
% Mention The epistemic limits of interpreting LLM (or human) behavior through psychosocial theories here or in the discussion section.
\section{Methods} \label{sec:Methods}
% ~1200 words ≈ ~9–10 paragraphs

\subsection{Data collection and preprocessing}
% This section can describe text processing steps - how the human responses were selected and which posts were selected from the RedArcs dataset etc.
A dataset was built from the RedditArchives for two public subreddits—AITAH, and Anxiety. For each subreddit, the top 1,000 most upvoted posts were selected, excluding weekly megathreads, deleted/removed items, and AutoModerator entries. For every retained post we extracted (i) the most upvoted comment and (ii) the comment that the OP engaged with most; all artifacts were saved to standardized CSVs for downstream analysis.

Text was cleaned with minimal, semantics-preserving preprocessing: we removed non-English items, de-identified obvious personal identifiers (usernames, emails, links), standardized whitespace and Unicode characters, and lightly constrained length (posts ~50–500 words; comments ~5–300 words) for comparability. We treat each Reddit thread (the post and its comments) as a single analytic unit during sampling, manual checks, and statistic aggregation, so correlated texts don’t inflate results. This preserves thread integrity and prevents dependence-induced bias when comparing human and LLM responses drawn from the same conversation.We also removed exact and near-duplicate texts (specifically, crossposts, copypastes and bot repeats) to prevent inflated counts and biased comparisons.

\smallskip Prompts for each step in the pipeline are provided in the appendices \ref{sec:prompts}

\subsection{Procedures}
% This section should discuss the creation of proxies for empathy and sycophancy.
For each selected post, we first prompt the target language model to generate an open-ended response to the body of the post. This response is appended to a table containing: (i) the model-generated response, (ii) the top upvoted human comment, and (iii) the most engaged human comment (available for approximately half of the posts). The resulting dataframe consists of the original post body, paired with two types of responses to personal queries - human and AI responses.

\medskip 
\textbf{Feature Extraction}

\smallskip Features are extracted at the sentence level, consisting of sentences from both the responses and post bodies that are annotated in accordance with steps 3 and 4 of the Evaluation Framework \ref{pipeline_steps}. Note that each feature is evaluated for \textbf{a.} \textbf{values exhibited} by the author and \textbf{b.} \textbf{values incentivized} by the author of the response. 


One of the central research questions (RQ2) investigates how the choice of psychosocial framework shapes the interpretation of an LLM’s response. Specifically, the same feature may be perceived as sycophantic under Goffman’s theory of face, empathic under Rogerian PCT, or as reflecting a terminal or instrumental value under Rokeach’s value framework.

To support this inquiry, the system constructs a dataframe that records: the original post, the set of extracted features for each of the 4 different types of responses (most-upvoted, most engaging, LLM 1, LLM 2) and the values or behaviors either exhibited or incentivized by each feature within any of the four applicable psychosocial paradigm(s).

This analytical dataset forms the basis for the subsequent analyses (see Section \ref{sec:Analysis}), where we analyze the differences in distributions of values in the responses obtained from reddit compared to the language model produced responses, within and across paradigms, to address RQ3.

\subsection{Experiments}
We conduct a series of experiments to investigate how psychosocial frameworks shape the interpretation of human and model-generated responses to personal queries. Our experimental design spans two dimensions: (i) response type (two forms of human responses and three language model responses) and (ii) domain (two distinct subreddits).

The AITAH dataset provides a natural proxy for “ground truth” in two paradigms: Empathy and Sycophancy. Here, crowd-sourced verdicts and their accompanying justifications offer a binary-valued reference point against which LLM behavior is evaluated.

\textbf{Experiment 1} evaluates the distributions of values and behaviors across the four response categories (human top-voted, human most-engaged, and two LLMs). We compare both the explicit values expressed by the respondent and the implicit values incentivized by the response under the four psychosocial paradigms - Rogerian PCT, Goffman’s ToF, Anthropic’s Value Tree and RVS.


The focus is on how these models differ in their coverage of values and behaviors relative to human responses. From the analyses obtained, we ascertain the occurrence and co-occurrences of values and behaviours in LLM and human responses to personal queries.


In \textbf{Experiment 2}, we evaluate how variations in prompt design influence the breadth of values expressed by the LLM. Specifically, we incorporate prompts that explicitly instruct the model to (i) generate a response most likely to be upvoted, and (ii) generate a response most likely to engage the author.


\subsubsection{Generations}
A set of targeted experiments are run with DeepReflect’s analyses to investigate the efficacy of control mechanisms to align the values in language model outputs more closely with those observed in human responses. The generation experiments are implemented using the following methods:

\begin{enumerate}
%     \item \textbf{With supervised fine-tuning (SFT)} [model: GPT-x, Paradigms: Empathy, Sycophancy]
% Experiments with Fine-tuning the language model on two synthetic datasets, generated to reflect (i) sycophantic and (ii) empathic behaviors.
% Additional experiments with temperature scaling for the Roger's PCT paradigm.

    \item \textbf{Chain-of-thought reasoning} [models: Claude; one of Qwen-3 or LLaMA-3.1; paradigms: Rogers PCT and RVS]
Prompt augmentation experiments, where values with low frequency in LLM responses are explicitly introduced and emphasized (e.g., Rogers PCT: Unconditional positive regard, Psychological freedom; RVS: A comfortable life).
\end{enumerate}

% \begin{enumerate}
%     \item \textbf{With supervised fine-tuning (SFT)}: We use the extracted distributions to generate synthetic responses that mirror human communication patterns, creating training data for supervised fine-tuning of language models on supportive communication.
    
%     \item \textbf{Chain-of-thought prompting}: The identified patterns inform chain-of-thought prompting techniques for GPT-4 and Claude, guiding these models to produce responses with appropriate levels of empathy, advice-giving, and self-disclosure for each community context.
% \end{enumerate}

\subsection{Construct Validity and Evaluation Metrics}
To assess construct validity, one human annotator labeled 100 randomly sampled post–response pairs across all four paradigms for each response type. The PCT framework encompasses 15 behaviors, Goffman’s ToF 5, the RVS 36, and Anthropic’s Value Tree 18.

Inter-rater reliability reached Cohen’s $\kappa$ above xx for all metrics, with an overall classification accuracy of yy. For the AITAH dataset, verdicts and accompanying statements in responses were used as proxies for Empathy and Sycophancy, each mapped onto five behaviors as defined by their respective theoretical traditions\footnote{This strategy is conceptually aligned with prior work on social sycophancy \cite{cheng-etal-sycophancy}}.


For the RVS and Anthropic Value Tree frameworks, which yield categorical distributions rather than binary judgments, pairwise error rates such as False Negative Rate (FNR) and False Positive Rate (FPR) are not directly applicable.
To identify significant associations between features annotated under more than one distinct paradigm we construct contingency tables and use chi-square analysis with further details provided in section~\ref{sec:Analysis}. 
% TODO: Add some math here for the chi-square analysis.


% Users of older versions of \LaTeX{} may encounter the following error during compilation: 
% \begin{quote}
% \tt\verb|\pdfendlink| ended up in different nesting level than \verb|\pdfstartlink|.
% \end{quote}
% This happens when pdf\LaTeX{} is used and a citation splits across a page boundary. The best way to fix this is to upgrade \LaTeX{} to 2018-12-01 or later.
% TODO: P6
% Results
% A no-nonsense report of what happened.
\section{Results}
% ~1600 words ≈ ~12–13 paragraphs
\textcolor{black!40}{A no-nonsense report of what happened.}
\subsection{Subsection}
This subsection presents the main results.

\textcolor{black!30}{\lipsum[40-44]}


\subsection{Subsection}
This subsection presents additional results and analysis.

\textcolor{black!30}{\lipsum[45-48]}

\subsection{Comparative Findings}
\textcolor{black!30}{\lipsum[49-52]}

% We encourage you to use the natbib styles.
% You can use the command \verb|\citet| (cite in text) to get ``author (year)'' citations, like this citation to a paper by \citet{Gusfield:97}.
% You can use the command \verb|\citep| (cite in parentheses) to get ``(author, year)'' citations \citep{Gusfield:97}.
% You can use the command \verb|\citealp| (alternative cite without parentheses) to get ``author, year'' citations, which is useful for using citations within parentheses (e.g. \citealp{Gusfield:97}).
% Analysis
% Discussion of what the results mean, what they don't mean, where they can be improved, etc. These sections vary a lot depending on the nature of the paper.
% For papers reporting on experiments with multiple datasets, it can be good to repeat Methods/Results/Analysis in separate (sub)sections for each dataset.
\section{Analysis}

\nocite{Ando2005,augenstein-etal-2016-stance,andrew2007scalable,rasooli-tetrault-2015,goodman-etal-2016-noise,harper-2014-learning}

The \LaTeX{} and Bib\TeX{} style files provided roughly follow the American Psychological Association format.
If your own bib file is named \texttt{custom.bib}, then placing the following before any appendices in your \LaTeX{} file will generate the references section for you:
\begin{quote}
\begin{verbatim}
\bibliographystyle{acl_natbib}
\bibliography{custom}
\end{verbatim}
\end{quote}
% You can obtain the complete ACL Anthology as a Bib\TeX{} file from \url{https://aclweb.org/anthology/anthology.bib.gz}.
% To include both the Anthology and your own .bib file, use the following instead of the above.
% \begin{quote}
% \begin{verbatim}
% \bibliographystyle{acl_natbib}
% \bibliography{anthology,custom}
% \end{verbatim}
% \end{quote}

\subsection{Interpretation of Results}
\lipsum[53-56]

\subsection{Theoretical Implications}
\lipsum[57-59]

\subsection{Subsection}
\lipsum[60-62]
% Quickly summarize what the paper did, and then chart out possible future directions that anyone might pursue. Finish with a strong conclusion.

\section{Conclusion}

\subsection{Summary of Findings}
\lipsum[63-64]

\subsection{Future Directions}
\lipsum[65]

% Scientific work published at ACL 2023 must comply with the ACL Ethics Policy.\footnote{\url{https://www.aclweb.org/portal/content/acl-code-ethics}} We encourage all authors to include an explicit ethics statement on the broader impact of the work, or other ethical considerations after the conclusion but before the references. The ethics statement will not count toward the page limit (8 pages for long, 4 pages for short papers).
% Limitations: Discuss the limitations of the paper as a complement to the discussion of strengths in the main text. This section should occur after the conclusion, but before the references.
\section*{Limitations}

ACL 2023 requires all submissions to have a section titled ``Limitations'', for discussing the limitations of the paper as a complement to the discussion of strengths in the main text. This section should occur after the conclusion, but before the references. It will not count towards the page limit.
The discussion of limitations is mandatory. Papers without a limitation section will be desk-rejected without review.

While we are open to different types of limitations, just mentioning that a set of results have been shown for English only probably does not reflect what we expect. 
Mentioning that the method works mostly for languages with limited morphology, like English, is a much better alternative.
In addition, limitations such as low scalability to long text, the requirement of large GPU resources, or other things that inspire crucial further investigation are welcome.
% Ethics Statement: Include an explicit ethics statement on the broader impact of the work, or other ethical considerations after the conclusion but before the references.
\section{Ethics Statement}

We encourage all authors to include an explicit ethics statement on the broader impact of the work, or other ethical considerations after the conclusion but before the references. 

The ethics statement will not count toward the page limit (8 pages for long, 4 pages for short papers).

\section*{Acknowledgements}
The authors would like to thank Santa Claus and Rudolph the red nose reindeer who had a very shiny nose. And if you ever saw it, you would even say it glows. All of the reindeer loved him, as they shouted out with glee, "Rudolph the red nose reindeer, you'll go down in history!"
% Entries for the entire Anthology, followed by custom entries
\bibliography{anthology,custom}
\bibliographystyle{acl_natbib}

\appendix

\section{Example Appendix}
\label{sec:appendix}

This is a section in the appendix.

\end{document}
